% Chapter 4 - Modalidad Háptica

\chapter{Modalidad H\'{a}ptica} % Chapter title

\label{ch:mod_haptic} 

%----------------------------------------------------------------------------------------

% ### Introducción de capitulo
La modalidad háptica es la que permite la interacción con el órgano mas grande del cuerpo humano, la piel. Particularmente, el sentido del tacto.
A través de este sentido, se pueden identificar dos interfaces, una cutánea y otra cinestésica. La primera se relaciona directamente con la piel y sus diferentes terminales sensitivas, la segunda, en cambio, con músculos y tendones y representa información sobre la ubicación espacial de todo nuestro cuerpo en el ambiente. Ambos canales proporcionan diferente \emph{feedback} al usuario y deberán ser capturados usando diferente hardware, \ie display táctil vs. display cinestésico. Ambos pueden trabajar en conjunto. 
Existen diferentes formas de estimular el sentido del tacto, los receptores hápticos del cuerpo humano son de tres tipos: mecano-receptores (actúan de acuerdo a la fuerza o presión), termo-receptores (estimulados por la temperatura que reciben) y nociceptores (estimulados por el dolor). En este trabajo se incluye un driver de modalidad que trabaja con displays táctiles y feedback vibro-táctil, por lo que se hará foco en la modalidad de los mecano-receptores.
