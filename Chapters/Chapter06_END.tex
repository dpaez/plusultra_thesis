% Chapter 6 - Conclusiones & Trabajo Futuro

\chapter{Conclusiones \& Trabajo Futuro} % Chapter title

\label{ch:end_future} % ### Introducción de capítulo
Conceptos como \emph{Wearable Computing}, \emph{Sensors Fusion} o incluso \emph{Pervasive Computing} han vuelto a ganar impulso recientemente. Utilizando tecnología como placas arduino o similares, implementar pruebas de concepto o prototipos de proyectos de las áreas antes mencionadas, se ha vuelvo mas simple, ya no es necesario contar con conocimientos avanzados de electrónica o saber programar un PIC; estas ventajas reducen las barreras para continuar experimentando en dichas áreas.

Es en este contexto de computación ubicua o ''fuertemente'' móvil es donde es interesante pensar en usar conceptos de desarrollo de aplicaciones multimodales, aprovechando el posible enriquecimiento en las formas de comunicación que estos nuevos ''caminos'' ofrecen para aumentar la experiencia del usuario dentro de la aplicación o incluso para introducir nuevos usuarios gracias a los beneficios conseguidos por la expansión en el conjunto de interacción del sistema.

\marginpar{Sharon Oviatt escribió en 1999 un trabajo donde ataca 10 mitos que suelen generarse alrededor de las aplicaciones multimodales \citep{oviatt1999ten}. Este trabajo resulta aun hoy en dia muy relevante y es una lectura recomendada para todo aquel que se vea frente a la necesidad de comenzar a desarrollar una aplicación multimodal e incluso para el público en general.}
De acuerdo a \citet{kortum2008hci}, los sistemas multimodales dan libertad a los usuarios de elegir como comunicarse con la aplicación, robustez para disminuir errores al tener mas de una modalidad para indicar una acción, \eg el comando <<apuntar aquí>> en una aplicación de información geoespacial, puede ganar precisión utilizando un disparador mediante voz (\emph{''apuntar aquí''}), sumado a un evento sobre una superficie táctil; especialmente si lo comparamos con una aplicación similar que solo cuente con una única modalidad (háptica o por voz) de interacción. Estos sistemas incluso pueden introducir ventajas de rendimiento para determinaras tareas (\eg  visual-espacial), aunque esto depende de la combinación de modalidades. Por ejemplo, en el caso de sistemas que combine gestos (mediante un lapiz optico) con ordenes via voz, resultados han mostrado que se obtiene una mejora del 10\% en los tiempos de terminación de la tarea, se reducen los errores críticos en un 36\% y una reducción del 50\% en problemas de fluidez espontanea, junto a mejoras en las construcción linguisticas, según \citet{oviatt1998referential}.

\section{Conclusiones Sobre el Desarrollo} \label{sec:end_work}
Dentro del marco del presente trabajo de grado, se ha desarrollado una plataforma que permite aumentar el conjunto de canales de interacción de una aplicación web, a través del soporte de interacciones multimodales. Estas nuevas capacidades de interacción pueden ser consumidas en conjunto o por separado, el desarrollador de la aplicación web es quien tiene el control sobre esto. Esta plataforma, en conjunto con el cliente y el protocolo desarrollados, han sido validadas mediante el desarrollo de una aplicación web que no solo permite su uso mediantes los canales clásicos como el teclado y mouse, si no que también es posible interactuar usando el canal háptico y gestual-aéreo. Incluso en el canal haptico se explora la retroalimentación, mediante el uso de motores de vibración. Es decir, el usuario puede \emph{sentir} nuestra aplicación. Todo esto sin perder las características clave de la web, como colaboración, fácil acceso y distribución.

Esta aplicación ha sido desarrollada con los fines de validar la plataforma propuesta y ayudar a mostrar las capacidades de los sistemas multimodales en la web. Por lo tanto constituye una simple demostración del alcance de la herramienta desarrollada. Con mas trabajo y recursos es perfectamente posible generar una experiencia mas rica que incluya, tal vez, a mas dispositivos y canales. 

Teniendo esto en cuenta, los avances aquí propuestos resultan satisfactorios y cumplen con el punto establecido en un principio, extender las capacidades de interacción de las aplicaciones web. La experiencia en general ha resultado satisfactoria y de gran aprendizaje, por ejemplo, en el desarrollo de una aplicación multimodal, uno de los componentes clave es el motor de fusión, en este trabajo se ha desarrollado un nuevo tipo de moto, si bien es bastante simple, cuenta con una característica completamente novedosa, esta distribuido entre los clientes. A través de mas pruebas seria posible saber que tan buena o mala puede ser este nuevo rasgo, en la aplicación a modo de demostración desarrollada, este motor ha sido funcional permitiendo una detección rápida de eventos mientras disminuía la complejidad en el desarrollo del mismo. En la próxima sección se detallan algunas de estas cuestiones, que pueden extenderse como posibles trabajos a futuro.

\section{Trabajo Futuro} \label{sec:end_future}
El desarrollo de un sistema nuevo en un escenario donde antes no existia nada similar de acuerdo a lo visto en el capítulo \ref{ch:estado_arte} supone un desafio. Mas alla de la dificultad per se de llevar a cabo el desarrollo, existen numerosas decisiones que deben ser tomadas con el objetivo de llegar a destino, algunas pueden ser supuestas con antelación y otras simplemente, no pueden serlo, mas aun cuando todo el equipo de trabajo esta constituido en una sola persona. Muchas de estas decisiones se encuentran detalladas en esta sección y pueden ser vistas como posibles mejoras a realizar, otras resultaron en nuevos problemas, que no podian ser tratados en su momento porque hubiesen desviado la atención y el objetivo principal del trabajo, construir la plataforma y aumentar las capacidades de interacción de las aplicaciones web; a las cuales se les ha brindado una solución que puede no ser la mas apropiada, pero que ha servido para cumplir el objetivo propuesto.

\subsection{Performance de los Motores de Fusión y Fisión Distribuidos}
Seria interesante conocer la performance de los motores de fusión y fisión desarrollados. Esto permitiría detectar problemas desconocidos o cuellos de botella de la plataforma. Además de brindar una forma de comparación con otras herramientas similares del mercado. Una posible técnica de medición de performance podría ser la propuesta por \citet{dumas2009benchmarking}. Durante el desarrollo del trabajo se priorizo cumplir con el objetivo de crear la plataforma, para luego de eso, considerar una etapa, en otro trabajo, de medición de performance.

\subsection{Mejoras Especificas al Desarrollo del Motor de Fusión}
Durante la creación del motor de fusión, uno de los momentos clave, fue el desarrollo de la ``ventana'' de captura de eventos. La misma es configurable, en cuanto al tiempo que se encuentra \textit{abierta} y fue construida usando algunas primitivas del lenguaje JavaScript. Construir un sistema de relojes propio, para controlar la apertura de la ventana seria ser una mejora notable al mantenimiento del sistema. También seria importante enriquecer el motor de fusión añadiendo soporte de prioridades entre modalidades, configurable por el desarrollador a su vez como el manejo de orden y no solo simple ocurrencia, de eventos.

\subsection{Mejoras de Usabilidad en la API}
Actualmente, la primer versión de la API del cliente es la que ha sido descripta en el capítulo \ref{ch:enlace}, y vista en uso en el capítulo \ref{ch:demo_development}. La misma puede ser mejorada, ya sea añadiendo nuevos métodos que hagan mas simple la tarea del desarrollador o removiendo o unificando operaciones existentes, con el mismo objetivo. Una manera de detectar estos puntos críticos es involucrando mas desarrolladores que consuman y utilicen la plataforma, teniendo en cuenta el feedback de los mismos en una nueva versión de la API.

\subsection{Consolidar a plusultra como PAAS}
La plataforma desarrollada tiene intenciones de funcionar como una plataforma como servicio (\emph{PAAS}). Es decir, podría convertirse en un producto comercial que brinde la capacidad de desarrollo web multimodal ocultando cuestiones de hardware externas al problema que se desea resolver. Para conseguir esto, algunas cuestiones fueron tenidas en cuenta durante el desarrollo de este trabajo, pero en otras se decidió no llevarlas a cabo, \eg capacidades de autenticación o un servicio web para indicar el desarrollo de una nueva aplicación multimodal.

\subsection{Crear Catalogo de Drivers de Modalidad}
Un aporte significativo para esta plataforma sería la creación de un catalogo de drivers de modalidad, ofrecido como un servicio web. De esta manera, cualquier persona que desee crear una nueva aplicación multimodal podría revisar primero el catalogo para saber si el hardware con el que cuenta tiene soporte. También serviría para impulsar y unificar el desarrollo de los mismos.

\subsection{Crear Aplicaciones Web usando la Plataforma}
Para detectar posibles problemas existentes que han permanecido en silencio o mejorar la usabilidad de la plataforma es importante y beneficioso contar con un ecosistema saludable de aplicaciones y drivers. El equipo de desarrollo detrás de una nueva aplicación multimodal que use la plataforma puede generar una considerable cantidad de feedback que puede ser muy útil.

\section{Enlaces a los Repositorios} \label{sec:end_links}
A continuación se encuentra un listado de enlaces a los repositorios de los componentes de la plataforma:
\begin{itemize}

\item \href{https://github.com/dpaez/plusultra}{Plataforma plusultra.}  (github.com/dpaez/plusultra)

\item \href{https://github.com/dpaez/gyes}{Módulo Interfaz gyes.} (github.com/dpaez/gyes)

\item \href{https://github.com/dpaez/HapticModalityDriver}{Driver de Modalidad háptico.} (github.com/dpaez/HapticModalityDriver)

\item \href{https://github.com/dpaez/AirPointerDriver}{Driver de Modalidad gestual-aéreo.} (github.com/dpaez/AirPointerDriver)

\item \href{https://github.com/dpaez/shapes\_app}{Aplicación Desarrollada usando la Plataforma.} (github.com/dpaez/shapes\_app)

\end{itemize}