% Abstract

\pdfbookmark[1]{Resumen}{Resumen} % Bookmark name visible in a PDF viewer

\begingroup
\let\clearpage\relax
\let\cleardoublepage\relax
\let\cleardoublepage\relax

\chapter*{Resumen} % Abstract name
En este trabajo de tesis se propone una herramienta que hace posible aprovechar mas de una modalidad dentro del contexto de una aplicación web, de forma que se aumente el conjunto de interacciones, soportando así nuevas interacciones multimodales.

Por interacciones multimodales, se hace referencia a todas aquellas interacciones, tanto para con el mundo físico real, como para el virtual a través de los diferentes \emph{modos}; cada uno de estos modos esta ``asociado'' a los sentidos del ser humano, de acuerdo a \citet{Bourguet2003}. Se busca, de esta forma, eliminar o disminuir barreras en la comunicación hombre-máquina/aplicación, \eg~ eliminando o dando una alternativa al uso del teclado, eligiendo favorecer a nuevas capacidades de interacción naturales, \ie~ asociados a los sentidos.

Cuando se permite operar con mas de uno de estos \emph{modos} en simultáneo, estamos frente a interacciones multimodales, una de las aplicaciones clásicas que abrió el camino multimodal, es el trabajo de Bolt \citep{bolt1980put} con \emph{put-that-there}, donde se estudiaba la interacción en una aplicación, de la voz y gestos para indicar comandos.

Las aplicaciones web son crecientemente populares, extendiéndose continuamente a nuevos dominios y plataformas. Un número increíblemente grande de usuarios entran en contacto directos con aplicaciones web diariamente utilizando dispositivos que ofrecen mecanismos de interacción diferentes, los cuales en muchos casos, pueden ser utilizados de manera simultánea.

Teniendo en cuenta este contexto, se ha desarrollado una plataforma, denominada \emph{Plusultra}, junto a un protocolo de comunicación y un cliente modular, \emph{Gyes}, que no solo permite consumir eventos generados por las modalidades si no también definir nuevos drivers para ampliar el catálogo de modalidades soportadas. Todas estas herramientas son \emph{open source} lo que significa que no solo pueden ser mejoradas por la comunidad si no también extendidas en alcance, por ejemplo, como se mencionó a través del desarrollo de nuevos drivers. Así es como se aumentan las capacidades interactivas de las aplicaciones web.

El contenido de este documento está organizado de la siguiente manera; el primer capítulo introduce al lector en el \emph{estado del arte} actual del soporte a interacciones multimodales. En el segundo y tercer capítulo se analiza en dos partes las estrategias elegidas para desarrollar la plataforma \emph{Plusultra} y el módulo cliente \emph{Gyes}. Luego, en el capítulo cuatro, se define que se entiende por aplicación web y como se extienden sus capacidades de interacción usando la plataforma. En el capítulo cinco, se muestra a todo el conjunto de tecnologías desarrolladas en acción, usando como ejemplo a la aplicación \emph{Shapes}, desarrollada específicamente para este trabajo. Finalmente, el lector encontrará en el capítulo seis, las conclusiones y las posibilidades de extensión, detalladas como \emph{trabajo a futuro}. A su vez se agregan dos apéndices con información en detalle sobre las modalidades elegidas en el trabajo, estas son háptica y gestual.

\endgroup			

\vfill